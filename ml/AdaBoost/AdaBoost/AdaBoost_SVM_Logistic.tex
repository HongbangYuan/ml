\documentclass{article}
\usepackage[UTF8]{ctex}
\usepackage{amsfonts, amsmath}
\usepackage{graphics, graphicx}
\usepackage{geometry}
\geometry{
    a4paper, 
    total={170mm, 257mm}, 
    top=20mm, 
    left=20mm 
}

\begin{document}

\begin{center}
    \textbf{SVM, AdaBoost, Logistics Regression学习策略与算法的比较}
\end{center}

\begin{center}
    \begin{tabular}{c c c c c}
        \hline
        学院 & 年级 & 班级 & 姓名 & 学号 \\
        \hline
        控制科学与工程 & 2017级 & 人工智能与机器人 & 陈逸群 & 201700181055 \\
        \hline
    \end{tabular}
\end{center}

\section{学习策略比较}
对于Logistic回归模型,其学习策略为:学习一个对数几率的线性回归模型,利用该模型来学习样本的生成概率分布,
属于生成模型。根据logistic分布,以二分类为例,给定输入数据x,其输出y的概率为
\begin{center}
    $$
    P(y=1|x) = \frac{e^{w\cdot x}}{1+e^{w\cdot x}}
    $$
    $$
    P(y=0|x) = \frac{1}{1+e^{w\cdot x}}
    $$
\end{center}
其中w是可学习参数。若一个事件的发生概率为p,将一个事件几率定义为$ \frac{p}{1-p} $,那么logistic回归
模型的对数几率为$ logit(p) = log(\frac{p}{1-p}) = w\cdot x $,为线性模型。\\\\
对于SVM模型,其学习策略为:对于一个线性可分的样本空间,学习到一个具有最大几何间隔的超平面将正负样本分开,
从而达到分类的目的。对于线性不可分的样本,可以采用核函数将其投影到非线性空间,对于噪声干扰大的数据,可以
采用带松弛变量于惩罚系数的支持向量机,从而学到能够使正负例样本的几何间隔最大的超平面。\\\\
对于AdaBoost模型,其学习策略为:通过训练一系列比随机预测略好的弱分类器,并根据训练时的分类误差率,为每一个
弱分类器分配一个权重,从而组成一个最终分类器,达到训练出一个强分类器的目标。
\section{算法比较}
对于Logistic回归模型,可以采用梯度下降法或者拟牛顿法来求解。\\\\
对于SVM,由于当样本容量变得很大的时候,通常的凸优化算法往往变得很低效,以至于无法使用,因此,
可以采用SMO (序列最小最优化算法),通过每一次选取两个需要更新的参数,逐步迭代直至满足KKT条件
以求得满意解。\\\\
对于AdaBoost,其求解算法为:在每一次迭代中,根据前一次的分类误差率对每一个训练样本分配不同的权重,
进而对当前的基本的弱分类器进行训练,同时根据每一次迭代的分类误差率,对该基本分类器分配相应的权重。

\end{document}